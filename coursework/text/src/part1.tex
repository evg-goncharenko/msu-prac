\chapter{Введение} \label{chapt1}


Широкое внедрение вычислительных машин во многие сферы человеческой деятельности вызывает необходимость изучения человеком специальных языков общения с машиной. Естественный язык не требует специального изучения, он обладает огромной выразительной мощностью и, следовательно, пользователю любой специальности и уровня подготовки будет удобно общаться с машиной на естественном языке. Поэтому проблема общения человека с ЭВМ на естественном языке является важной практической задачей. \\
Достаточно вспомнить, как человек общался с первыми вычислительными машинами. А происходило это таким образом: оператор брал в руки провода с двумя разъемами на конце и соединял между собой триггеры, из которых собственно и состояла машина, таким образом, чтобы при запуске выполнялась нужная последовательность команд. Внешне это очень напоминало манипуляции телефонных барышень начала века. А по сути – это была очень квалифицированная работа. Программирование осуществлялось даже не в машинных командах, а на аппаратном уровне. Соответственно и квалификация тогдашних программистов была очень высокого уровня. \\
Потом задача упростилась, последовательность нужных команд стала записываться непосредственно в память машины, а для ее ввода стали применяться более скоростные и производительные устройства. Затем появились перфокарты, а чуть позже перфоленты. Скорость общения с машиной возросла. Количество ошибок при вводе резко уменьшилось, но сущность этого общения, его характер не изменились. \\
Сам процесс общения с машиной долгое время оставался уделом специалистов, недоступным для понимания простыми пользователям. Теми самыми "простыми пользователями", которые, собственно говоря, и являлись потребителями компьютерных услуг. Компьютерный интерфейс на первых этапах развития компьютерной техники в качестве обязательного элемента непременно включал в себя человека-специалиста. Вот если бы можно было бы пообщаться с компьютером напрямую, да при этом еще и не загружать свой багаж знаний всяческими техническими компьютерными сведениями... \\
Различные системы используют автоматическую обработку текстов. Интеллектуальные «вопрос - ответные» системы характеризуются конкретной предметной областью, разбор высказываний на естественном языке производится по некоторым правилам разбиения, а синтез ответа по некоторым правилам сборки. Поскольку в такой системе нет лингвистической обработки текста, ответы системы не всегда корректны. В системах общения с базами данных высказывания на естественном языке переводятся в запрос на формальном языке. В этих системах тоже нет лингвистического анализа текста. В диалоговых системах решения задач помимо перевода высказываний в формальное внутреннее представление решается некоторый достаточно узкий класс задач (например, планирование путешествий или составление контрактов). Так же автоматическая обработка текста используется в обучающих системах, системах распознавания речи, в системах машинного перевода, при создании редакторов текстов для выявления ошибок, связанных с порядком и сочетаемостью слов в предложении, в информационно-поисковых системах, а также для автоматизации лингвистических исследований. \\
Создавались системы обеспечивающие диалог с машиной на так называемом «ограниченном» естественном языке. Это системы ПОЭТ (программа обработки экономических текстов) Э. В. Попова, ДИЛОС В. М. Брябрина, ДИСПУТ Л. И. Микулича и А. Я. Червоненкиса. Но использование ограниченного естественного языка не так удобно для пользователя, как общение с ЭВМ на действительно естественном языке. Ограниченный естественный язык требует предварительного изучения принятых в системе ограничений на словарь и грамматику. Это затрудняет процесс общения пользователя с машиной. Часто пользователю проще выучить некоторый формальный язык и общаться с машиной на нем, чем постоянно следить за соблюдением правил наложенных на естественный язык. В этих системах также нет средств для исследования незнакомых слов и структур предложений, запоминания новых фактов для последующего их использования. \\
Среди систем, имитирующих понимание естественного языка, известны APRIL, TULISP, TULISP-2 М. Г. Мальковского. Программа APRIL решает довольно узкий класс задач из школьного курса математики. Каждая задача решается независимо от других, никакой информации о решении программа не запоминает. В программе применяется достаточно полный и глубокий лексический и синтактико-семантический анализ фраз с использованием словаря, распознающей контекстно-свободной грамматики и описания, значимых рассматриваемой предметной области, семантических конструкций. Производится тщательный анализ предложений, необходимых для сведения задачи к одному из допустимых типов, и условий, определяющих существование решения и применимость допустимых типов, и условий, определяющих существование решения и применимость одного из стандартных способов решения. В то же время возможности учета в процессе анализа проблемно-ориентированной информации о характерных задачах были использованы далеко не в полной мере. Программа TULISP является реализованной на программном уровне диалоговой системой искусственного интеллекта широкой ориентации. Имеет три предметные области: решение арифметических задач в словесной формулировке (модернизированный вариант решателя APRIL), решение задач на планирование действий пользователя (на основе метода редукции задач), обучение языку (адаптация языковых знаний TULISP с привлечением данных из метаграмматики). Помимо этого имеется блок, осуществляющий синтез, адресуемых пользователю, сообщений, соответствующих запросам, комментариям, описаниям полученных результатов. Достаточно детально был разработан и внутренний язык программы - язык представления знаний. Развитие методов адаптации и обучения, более полное описание русского языка, разработанное на основе лингвистических данных, средства обнаружения и исправления речевых ошибок пользователя, появление режима, в котором с помощью специальных директив можно контролировать текущее состояние языковых знаний системы (словарей, грамматик) привело к появлению новой версии системы TULISP – системы TULISP-2. \\
Задача интеллектуальной обработки текстов на естественном языке впервые появилась на рубеже 60х—70х гг. С тех пор было предпринято множество различных попыток ее решения [2-6], созданы десятки экспериментальных программ, способных вести диалог с пользователем на естественном языке. Однако широкого распространения такие системы пока не получили — как правило, из-за невысокого качества распознавания фраз, жестких требований к синтаксису “естественного языка”, а также больших затрат машинного времени и ресурсов, необходимых для их работы. Практически во всех системах машинного понимания текста используется ограниченный естественный язык, поскольку полной и строгой формальной модели ни для одного естественного языка пока не создано. \\
Продолжаются исследования в рамках программы создания информационных систем, в том числе и выполняющих обработку текстов на естественном языке. Один из классов таких систем образуют информационно-поисковые системы (ИПС), ориентированные на естественно-языковое общение с пользователем. Подобные ИПС могут использоваться в качестве консультанта — например, в области законодательства, медицины или любой другой предметной области, для которой характерно наличие большого количества информации, представленной документами на естественном языке. \\
Одним из ключевых элементов ИПС с естественно-языковой ориентацией также является лингвистический процессор, выполняющий роль посредника между пользователем и базой данных, в которой хранится интересующая его информация. Он включает в себя формальную модель естественного языка, базу данных модели (словарь) морфологический, синтаксический, семантический компоненты, кроме того может быть добавлена компонента, связанная с автоматической расстановкой ударений. \\
Задачей лингвистического процессора является преобразование естественно-языкового предложения (или даже целого текста) в некоторый набор семантических структур, являющихся формальным представлением “смысла” исходного предложения или текста. Цель такого преобразования — обеспечить исходные данные для работы поисковых механизмов СУБД. 

Задачей лингвистического процессора является преобразование \\
естественно-языкового предложения (или даже целого текста) в некоторый набор семантических структур, являющихся формальным представлением “смысла” исходного предложения или текста. Цель такого преобразования — обеспечить исходные данные для работы поисковых механизмов СУБД.

\newpage

Вот список тех задач, в которых можно использовать лингвистический процессор: 
\begin{enumerate} 
    \item написание переводчика;
    \item задачи распознавания и синтеза речи;
    \item распознавание текста;
    \item проверка орфографии;
    \item проверка синтаксиса;
    \item информационно-поисковые системы;
\end{enumerate}

Основным недостатком существующих лингвистических процессоров является чрезмерно большой объем словаря, порождающий ряд технических проблем:
\begin{itemize}
    \item большие затраты труда на создание и поддержание словаря;
    \item невозможность полного размещения словаря в оперативной памяти компьютера при анализе;
    \item высокая избыточность информации, связанной с постоянными  \\ признаками каждой словоформы (морфологическими, синтаксическими, семантическими);
\end{itemize}

Современные компьютерные программы, анализирующие текст на естественном языке, как правило, используют словари. Цель словарей помочь распознать встреченную текстовую цепочку.

Целью данной работы является создание программы, которая, используя реально существующую лингвистическую базу данных, выдает морфологические характеристики некоторых слов, не содержащихся в этой базе данных.
Программа основана на использовании морфологического анализа структуры незнакомого слова.
Приблизительно анализ слова работает в такой последовательности. От предлагаемого слова отрезаются возможные префиксы, и оставшаяся часть проверяется на наличие в лингвистической базе данных. Если оставшаяся часть слова присутствует в базе данных, то в качестве информации об исходном слове, выдается полученная информация о части слова, с учетом всех префиксов.

\clearpage