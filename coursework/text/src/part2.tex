\chapter{Постановка задачи} \label{chapt2}

Целью данной работы является создание программы, которая, используя реально существующую лингвистическую базу данных, выдает морфологические характеристики для следующих классов слов:
\begin{itemize}
    \item свободнообразуемые слова;
    \item слова с дефисом;
    \item сложные слова.
\end{itemize}

Свободнообразуемые слова должны удовлетворять следующим условиям:
\begin{itemize}
    \item Стандартность их соединения с существительными и прилагательными.
    \item Стандартность значения.
    \item Структурная самостоятельность.
\end{itemize}

Программа работает следующим образом: \\
От предлагаемого слова отрезаются возможные префиксы, и оставшаяся часть проверяется на наличие в лингвистической базе данных. Если оставшаяся часть слова присутствует в базе данных, то в качестве информации об исходном слове, выдается полученная информация (падеж, склонение и т.д.) о части слова, с учетом всех префиксов. Программа автоматически меняет основу, ударную букву, ставит второстепенное ударение. 

\clearpage