\chapter{Алгоритм анализа слова} \label{chapt3}

Анализ слова сводится к следующей последовательности действий:

\begin{enumerate} 
    \item На вход процедуры подается слово Х.
    \item Пытаемся найти в слове Х дефис. Если мы его нашли, то ту часть слова Х, которая была до дефиса (включая дефис), мы сохраняем как Х1; а оставшуюся часть слова Х как Х2 и переходим к шагу 5, иначе на шаг 3.
    \item Если слово Х начинается на префикс из списка префиксов (см. пункт: Словарная информация для базы данных), то мы сохраняем этот префикс как Х1, а оставшуюся часть слова Х как Х2 и переходим к шагу 5, иначе на шаг 4.
    \item Если слово Х начинается на порядковое числительное (например: тысячетрехсотдвадцатичетырехдневный), то мы сохраняем это \\числительное как Х1, а оставшуюся часть слова Х как Х2 и переходим к шагу 5, иначе на шаг 8.
    \item Обращаемся к морфологическому анализатору со словом Х2. Если морфологический анализатор выдал морфологические характеристики слова Х2, то перейти на шаг 6. Если же морфологический анализатор выдал, что слово не найдено, то перейти на шаг 7.
    \item В качестве информации о слове Х, выдается информация о слове Х2, модифицированная следующим образом:
    \begin{itemize}
        \item к основе слова Х2 слева приписывается слово Х1;
    \end{itemize}
    \begin{itemize}
        \item меняется номер ударной буквы;
    \end{itemize}
    \begin{itemize}
        \item ставится второстепенное ударение.
    \end{itemize}
    \item Рекурсивно вызываем данный алгоритм для слова Х2. Если алгоритм выдал морфологические характеристики слова Х2, то перейти на шаг 6. Если же алгоритм выдал, что слово не найдено, то перейти на шаг 8.
    \item Слово не распознано. Стоп.
\end{enumerate}

\clearpage